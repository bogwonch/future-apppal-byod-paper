\documentclass[a4paper]{article}

\usepackage{microtype}
\usepackage{booktabs}
\usepackage{hyperref}
\usepackage{acronym}
\acrodef{MDM}{mobile device management}
\acrodef{BYOD}{bring your own device}
\acrodef{HiMSS}{Healthcare Information and Management Systems Society}

\usepackage{algorithm}
\usepackage{algpseudocode}

% \apppal{} and Listing formatting
\usepackage{listings}
\lstdefinelanguage{AppPAL}{%
  morekeywords={if,inf,says,where,true,false},
  otherkeywords={can-act-as,can-say},
  sensitive=true,
  morestring=[b]',
  literate={\ inf\ }{{$\infty$}}5
}[keywords,strings]
\lstset{%
  basicstyle=\ttfamily\footnotesize{},
  stringstyle=\sffamily\footnotesize{},
  keywordstyle=\ttfamily\slshape\footnotesize{},
  language=AppPAL,
  columns=flexible,
  frame=single,
  framesep=0.5pt,
  framexleftmargin=2.5pt,
  framexrightmargin=2.5pt,
}

\newcommand{\dotdotdot}[1]{#1}
\newcommand{\etal}[0]{et~al{.}\@}
\newcommand{\apppal}[0]{App{P}{A}{L}}

\title{Case Study Reasoning about BYOD Policies in the Healthcare Sector}
\author{Joseph Hallett and David Aspinall}
\begin{document}
\maketitle

% 4 sentences
\begin{abstract}
  Implementing a \ac{BYOD} policy can be difficult especially in hospitals, where there is a need to balance privacy requirements against the need to treat patients.
  It is common to publish a \ac{BYOD} policy to describe how staff should use their devices in the workplace.
  These policies are written using natural language, and employees are expected to read and follow them: this can be tricky as the policies are often poorly specified.
  We show how \apppal{}, a policy language for mobile app preferences, can be used as a tool for reasoning about policies and how it found issues with BYOD policies used in hospitals and as guidelines.
\end{abstract}

\section{Introduction}
% 1 page
\label{sec:intro}
% 1. Describe the problem
%    - BYOD hard
%    - Hospitals are an interesting case
%    - Need tools for reasoning about policies not enforcing

Employees bringing their own devices to work has become the norm:
  around 70\% of all companies having some form of \ac{BYOD} scheme~\cite{schulze_byod_2016}.
In hospitals the use of personal devices as part of patient care is particularly common.
One survey from the UK found that around 80\% of surgical doctors were willing to use their personal devices for work, with 85\% regularly using them to look up information, and 50\% having medical apps already installed on their phones~\cite{patel_uk_2015}.  Another from the US found similar numbers of physicians regularly using their devices as part of their work~\cite{moyer_managing_2013}.

Several healthcare organisations have started looking into using mobile devices as part of patient care, and found that these devices may be helpful~\cite{prgomet_impact_2009,faulds_feasibility_2016}.  
Organisations, such as the US's \ac{HiMSS}~\cite{seymour_mobile_2012}, have published guidelines helping hospital administrators implement BYOD policies in their workplaces.  
Additionally general advice for other organisations seeking to implement BYOD policies is available from SANS~\cite{nicholas_r._c._guerin_security_2008}, NIST~\cite{souppaya_guidelines_????}, HP~\cite{hp_byod_????} and the CESG~\cite{cesg_byod_2015} amongst others.
The NHS has even published guidelines for developers seeking to develop apps for use in hospitals~\cite{prgomet_impact_2009}.

Whilst there has been a push for bring \ac{BYOD} and mobile devices into the healthcare industry there has been push back against having mandated security policies automatically enforced.  
The use of \ac{MDM} software to enforce policies can be problematic:
  a survey found that even for companies that use \ac{MDM}, over half of them still have devices in their networks that do not meet their policies~\cite{mobileiron_security_labs_q4_2015}.
In the healthcare industry the circumvention of security controls by physicians is normal, and in some cases taught as good practice~\cite{koppel_workarounds_2015}.
The use of tools to automatically enforce policies in a hospital environment is not always the right approach:
\emph{Moyer} concluded that:
\emph{``Since mobile devices are already being used in hospitals, an approach toward educating the users instead of trying to control the technology may be more practical in the short term''}~\cite{moyer_managing_2013}.
Similarly \emph{Faulds}~\etal~echo this sentiment noting that \emph{``concerns about data protection are legitimate''}, and going on to \emph{``encourage the Information Commissioner’s Office, in consultation with professional medical organisations, to publish guidance specific to the use of BYOD technology''}~\cite{faulds_feasibility_2016}.

The conflict between patient care and data security, as well as \ac{MDM} software being relatively limited, has lead to \ac{BYOD} policies being published as natural language documents that employees are supposed to agree to follow.
Using natural language in policies can create ambiguities, and lead to poorly specified rules.
If a policy is poorly specified and confusing then its subjects may chose to ignore it or misinterpret it's rules.
If we wish to improve our policies we need to be able to reason about them clearly.
We need to be able to identify problems before they happen.

% 2. State your contribution
%    - Standard lexicon of predicates and idioms
%    - Formalisation of several BYOD policies
%    - Identification of problems in said policies
%    - Methods for inferring when \apppal{} policies are problematic

\apppal{} is a policy language for describing mobile app preferences~\cite{hallett_apppal_2016}.
We show how \apppal{} can also be used to formalise \ac{BYOD} policies by translating three \ac{BYOD} policies into \apppal{}.
One is taken from an NHS hospital trust~\cite{kennington_mobiles_2014}. 
Another is from \ac{HiMSS} published as an example policy to help hospitals design their own policies~\cite{healthcare_information_and_management_systems_society_mobile_2012}.
The third is a  general policy from SANS to help organisations to develop their own \ac{BYOD} schemes~\cite{nicholas_r._c._guerin_security_2008}.
Using these policies we suggest a standard lexicon of BYOD predicates and idioms for using when reasoning about \ac{BYOD} policies. 
With the AppPAL formalisation we show how problems with reachability and redundancy can be identified automatically, and suggest where the policies could be improved.

%\section{My Idea}
\section{Formalising policies using AppPAL}
% 2 pages
\label{sec:idea}

Each of the three policies we have chosen to look at has a different style in expressing their rules.
Despite the different styles, however, there are areas where the policies overlap and make similar rules.
We present the policies here as excerpts alongside their AppPAL translations.
The full translations and original policies can be found online\footnote{\url{https://github.com/apppal/apppal-byod-policy-translations}}.

The \ac{HiMSS} policy is taken from their \emph{Mobile Security Toolkit} for assisting healthcare organisations and security practitioners in managing the security of their mobile computing devices. 
\ac{HiMSS} itself is an American non-profit company advising healthcare companies on IT.
Their policy~\cite{healthcare_information_and_management_systems_society_mobile_2012} takes the form of a short user agreement that it's subjects are expected to sign.  
This gives a different style to the other policies: policies are expressed from the perspective of the subject, rather than as decrees from the policy authors.
It also contains many rules where the subject is expected to acknowledge a risk or regulation.  
For example in one rule the subject is required to acknowledge that networks can be unreliable:
\begin{lstlisting}[title={\footnotesize\textbf{\ac{HiMSS}}:~\itshape 
I understand and accept that synchronisation relies on one or more cellular network providers and the Internet, and that both are subject to slowdowns and outages of extended duration that are beyond the control of IT.}]
'ac{HiMSS}' says User can-say User hasAcknowledged('internet-unreliable')
  if User isUser.

'ac{HiMSS}' says User hasAgreeedToPolicy
  if ...,  User hasAcknowledged('internet-unreliable').
\end{lstlisting}
These rules are interesting as on the face of them they do not say anything about security or how the subject should use their device.
We translate rules like this into acknowledgement statements in AppPAL, which can be collected into a check that the user has acknowledged everything they should have done.

The NHS policy~\cite{kennington_mobiles_2014}, in contrast, is a more prescriptive document.
It is taken from the \emph{Torbay and Southern Devon NHS trust}\footnote{The NHS is organised into a series of \emph{trusts}, each of which contains several hospitals.  A doctor may work at several hospitals within a trust.}. 
It is interesting in part because it represents a \ac{BYOD} policy in actual use, but also because it is significantly more complex than most \ac{BYOD} policies as it goes beyond basic requirements such as requiring encryption by describing relationships between organisational groups within the trust and precise organisation specific requirements.
Like the \ac{HiMSS} policy it has rules that just require an acknowledgement (for example Rule~3.2 requires that the device owner \emph{``accepts that the device can be used to communicate through all corporate channels''}) but there is a greater emphasis on the specifying the responsibilities of different groups and roles within the hospital.  For example in Rule~2.4 a staff member who has applied for a phone needs it to be signed off by a finance director and a relevant assistant director:
\begin{lstlisting}[title={\footnotesize\textbf{NHS}:~\itshape 
Upon receipt of the completed application form, allocation will also need to be authorised by the relevant Assistant Director and the Director of Finance.}]
'nhs-trust' says Staff hasAppliedForPhone(Form)
  if Staff hasSubmitted(Form),
     'assistant-director' hasAuthorized(Form),
     'finance-director' hasAuthorized(Form).

'nhs-trust' says X can-say 'assistant-director' hasAuthorized(Form)
  if X can-act-as 'assistant-director',
     Staff hasSubmitted(Form),
     X isManagerOf(Staff).
\end{lstlisting}

The SANS Institute is a private US company that specialise in information security.
Whilst their example \ac{BYOD} policy is not healthcare specific, it is still interesting as there advice is widely regarded, and contains more general security advice.
The SANS \ac{BYOD} policy is split into sections that describe how device features, such as Bluetooth or OTA provisioning, should be configured.  Some rules are acknowledgements: for example the use of internet services in Section~2.8 requires acknowledgement of the email security, acceptable use and an eCommerce security policy.
Many rules describe when device features should be disabled, in contrast to the other two policies which tend to favor describing when features can be used.
For example in Section~2.4 the SANS policy states that infrared support should be disabled if Bluetooth is available: this is easily expressed in AppPAL:
\begin{lstlisting}[title={\footnotesize\textbf{SANS}:~\itshape 
Infrared supported MUST be disabled if Bluetooth connectivity is supported.}]
'sans' says Device shouldDisable('irda')
  if Device hasFeature('bluetooth').
\end{lstlisting}

% - Brief intro to apppal but mostly link to existing paper
%   - Why apppal?
% - Give an overview of the different policies we're looking at
% - Show differences and similarities between the policies
% - Make sure to tie it all into healthcare, but also be clear this is more general

\subsection{Why AppPAL?}

AppPAL is an authorization language designed to model app preferences~\cite{hallett_apppal_2016}.
It was created by instantiating the SecPAL language (which was designed for access control in distributed computing environments~\cite{becker_secpal:_2010}) with predicates and checking functions for describing apps.
It supports delegation, roles and constraints including arbitrary external checking programs to describe precisely what apps users might want to use.

AppPAL (and SecPAL) is designed to be readable.
It is powerful enough to describe complex policies and rules, whilst also being semantically simple having only three rules for evaluation.
When reasoning about \ac{BYOD} policies, it simplicity and readability gives it an advantage over more powerful languages such as XACML~\cite{oasis_extensible_2013} as a rule written in AppPAL can be comprehended quickly.
Whilst an end-user might not write a policy in a formal language, an IT administrator might wish to use a formal language to aid with reasoning and checking the policies they write.
AppPAL gives these IT professionals an easy entry point to formal methods, without over complicating things.
Early experiments showed that AppPAL can describe \ac{BYOD} policies, and help make clarify mistakes ahead of time~\cite{hallett_specifying_2016}.

An AppPAL evaluation library has been developed that runs on Java and Android and is available online\footnote{\url{https://github.com/apppal/libapppal}}.
As part of this work we have developed (in Section~\ref{sec:details}) a series of policy analysis tools for AppPAL.
These are compiled alongside the main AppPAL interpreter and may be run from the command-line.

\section{The Details}
% 5 pages
\label{sec:details}

% - Show the standard predicates and explain how they help to give structure to apppal policies
%   - show how they are tied into the trust relationships with patient care (i.e. consent)
% - Show the consistency checker
%   - Give algorithm and intuition
%   - Show examples where it has found problems
% - Show redundancy checks
%   - Algorithm and intuition
%   - Examples where problems were found

\subsection{Policy Idioms}

When translating the \ac{BYOD} policies we found types of actions or checks being performed.
There are checks if things are of a particular type (for~example in the NHS policy rule~2.1  a staff member is elligable for a mobile device if they are lone-working, work out-of-hours and work out-of-office).  
There are checks if an action is allowed (such~as whether they can connect to an wireless access point in the \ac{HiMSS} policy);  and checks to see if an action has already been performed (like the need to have acknowledged a policy in all three policies).
We also find policy rules that state the user should perform an action (such as disabling phone functionality in the SANS policy, or an obligation to report a phone missing in the NHS policy).

To help structure the formalisation we give the AppPAL predicates used to translate the policies prefixes.
The prefixes help to clarify the meaning of each predicate and groups the predicates into four categories.

\begin{tabular}{l l}
  \toprule
  Predicate & Meaning \\
  \midrule
  \textit{Subject} \texttt{can}\textit{Action}  & The \emph{subject} is allowed to perform the \emph{action}. \\
  \textit{Subject} \texttt{has}\textit{Action}  & The \emph{subject} has performed the \emph{action}. \\
  \textit{Subject} \texttt{is}\textit{Propery}  & The \emph{property} is true about the \emph{subject}. \\
  \textit{Subject} \texttt{must}\textit{Action} & The \emph{subject} is obliged to perform the action. \\
  \bottomrule \\
\end{tabular}

\subsubsection{Acknowledgements}

A common idiom in each of the policies is requiring an acknowledgement from the device owner that a policy exists, or that the company may have certain rights.  
These rules are interesting as there is often not a technical means to enforce them.
For example both the NHS and HiMSS policies have rules stating that they cannot be held liable if a user loses personal information if their device needs to be erased to protect personal information.

\begin{lstlisting}[title={\footnotesize\textbf{\ac{HiMSS}}:~\itshape 
I agree to hold XYZ Health System harmless for any loss relating to the administration of PDA/Smartphone connectivity to XYZ Health System systems including, but not limited to, loss of personal information stored on a PDA/Smartphone due to data deletion done to protect sensitive information related to XYZ Health System, its patients, members or partners.}]
'user' says 'user' hasAcknowledged('data-loss-policy').
\end{lstlisting}

\begin{lstlisting}[title={\footnotesize\textbf{NHS}:~\itshape 
Individuals who have personal data of any kind stored on a corporately issued mobile device must be aware that in the event of loss of the device the above data wipe will include removal of all personal data.}]
'nhs-trust' says Staff can-say
	Staff hasAcknowledged('data-loss-policy') 
	if Staff isStaff.
\end{lstlisting}

This idiom is particularly common in the \ac{HiMSS} policy where over the rules require some level of acknowledgement, and a third require an acknowledgement alone (\autoref{tab:idioms}).
Similarly in the longer NHS policy almost a quarter of the rules require the subject to acknowledge a rule.
In contrast the more technically focussed SANS policy has only one occurance of the idiom, when describing the other policies a user should be following in order to use a mobile device.

\begin{lstlisting}[title={\footnotesize\textbf{\ac{SANS}}:~\itshape 
Users MUST agree to the email security/acceptable use policy and eventually to the eCommerce security policy.}]
'company' says User canUse(Device) 
  if User hasDevice(Device),
     User hasAcknowledged('email-security'),
     User hasAcknowledged('acceptable-use'), 
     User hasAcknowledged('ecommerce-security').
\end{lstlisting}

The reliance in both the NHS and \ac{HiMSS} policies suggests that \ac{BYOD} policies often go beyond technical rules; something ignored in previous work on policy formalisation.
These policies often require that employees \emph{follow the rules}, and have an awareness of the systems surrounding them.

\begin{table}\centering
  \begin{tabular}{l c c c}
    \toprule
    Idiom                    & \ac{HiMSS}   & NHS      & SANS    \\
    \midrule
    Acknowledgement          & 5 (33\%) & 9 (18\%) & 1 (3\%) \\
    Partial Acknowledgement  & 3 (20\%) & 2 (4\%)  & 0       \\
    \bottomrule             \\
  \end{tabular}
  \label{tab:idioms}
  \caption{Occurences of policy idioms in each of the \ac{BYOD} policies.}
\end{table}


\subsection{Checking Reachability}

Inference in AppPAL happens by collecting ground facts and constraints that satisfy rules.
These rules are then combined to form a policy. 
If a rule is reachable then we would expect there to be some combination of facts that could satisfy the rule.
If there are no facts that could satisfy the rule then this suggests that the policy may be incomplete as there are policy rules that can never be satisfied.  

We search for reachable assertions using the algorithm in Listing~\ref{alg:reachable}:
it produces a set of pairs of predicates and speakers where the a pair of a speaker and a predicate indicates that that speaker may say something about that predicate.
We search over all the assertions in the AppPAL assertion context.
If all of an assertion's conditionals (the facts in the if part) are reachable (or it has none) then the speaker and predicate are added to the reachable set.
If the statement is a can-say statement then we additionally check if the delegated predicate is reachable from the delegated speaker, and if so mark the delegated statement as reachable from the speaker who made the can-say statement.

\begin{lstlisting}[language=Python,float,caption={Procedure for finding all reachable assertions.},label={alg:reachable}]
def reachable(ac) -> set:
  reachable = new set()
  iterate = True
  while iterate == True:
    iterate = False
    for assertion in ac:
      e = a.speaker
      p = a.predicate
      if p.isCanSay() and (e, p) in reachable:
        if (p.delegator, p.delegation) in reachable:
          if for all c in a.conditions: (e, c.predicate) in reachable:
            reachable.add((e, p.delegation))
            iterate = True
      else if not (e, p) in reachable:
        if for all c in a.conditions; (e, c.predicate) in reachable:
          reachable.add((e, p))
          iterate = True
  return reachable
\end{lstlisting}

\subsection{Checking Redundancy}

To check redundancy we convert the policy into a graph structure.
Each statement we might want to prove (a goal) becomes a node in the graph, its children are organised into sets of goals (a proof), where if all the goals in any set were proved true then the goal node would also become true.
If a node has an empty set of goals to prove it is a fact.
Once the graph has been constructed (taking into account delegation and unification with other statements), the graph is flattened by applying the flatten procedure (Listing~\ref{alg:flatten}) repeatedly until a fixed point is reached.
When the graph has been flattened we look for redundancy by looking for goals with proofs that are subsets of their other proofs, implying that if the larger proof is true, then the shorter proof will always be true too; 
  and different goals with identical proofs, implying that the two decisions are not independent~(Listing~\ref{alg:redundancy}).

\begin{lstlisting}[language=Python,float,caption={Procedure for flattening the redundancy graph.},label={alg:flatten}]
def flatten(graph):
  for goal in graph.goals:
    hoist = true
    for proof in graph[goal]:
      for proof_goal in proof:
        if not proof_goal.is_fact:
          hoist = False
          break
    if hoist == True:
      hoist(graph, goal)

def hoist(graph, target):
  for parent in graph[target].parents:
    for proof in parent.proofs:
      if proof.uses(target):
        for replacements in graph[target].parents:
          parent.proofs += proof.replace(target, replacements)
        parent.proofs -= proof
\end{lstlisting}

\begin{lstlisting}[language=Python,float,caption={Procedure to check for redundancy.},label={alg:redundancy}]
def check_redundancy(graph) -> boolean:
  for goal in graph.goals:
    for a in graph[goal]:
      for b in graph[goal]:
        if a >= b and a.goals.subset(b.goals):
          if b.goals.subset(a.goals):
            warn(a+" has multiple equivalent proofs") 
          else
            warn(a+" has a redundant proof")
    for other_goal in graph.goals:
      if goal > other_goal:
        for a in graph[goal]:
          for b in graph[other_goal]:
            if not (a.goals.is_empty() or b.is_empty())
               and a.goals == b.goals:
              warn(a+" and "+b+" have equivalent proofs")
\end{lstlisting}


\section{Related Work}
% 1-2 pages
\label{sec:related}

Armando~\etal~created a BYODroid~\cite{armando_bring_2013}; \ac{BYOD} framework to enforce \ac{BYOD} policies by extending Featherweight Java~\cite{igarashi_featherweight_2001} with a type system that can describe how Android apps interact.  Whilst this work gives a framework for enforcing policies on Android, it doesn't help policy authors decide what the policies should be to begin with.

\section{Conclusions}
% 0.5 pages
\label{sec:conclusions}

\bibliography{paper}{}
\bibliographystyle{plain}
\end{document}

