\documentclass{article}

\usepackage{microtype}

\usepackage{acronym}
\acrodef{MDM}{Mobile Device Management}

\title{Capturing Policies for BYOD}
\author{Joseph Hallett and David Aspinall}
\begin{document}
\maketitle

\begin{abstract}
  % 4 Sentences
\end{abstract}

\section{Introduction}
\label{sec:intro}
% Describe the problem
% State my contribution
% 1 page

Employees bring their own devices to work.  
In the past employees might have had a dedicated company device. 
Today around 70\% of companies have a BYOD scheme~\cite{schulze_byod_2016} and, in some fields,
  85\% of staff use their personal devices to look up work-sensitive information~\cite{patel_uk_2015}. 
This creates a challenge for IT departments.
They need to control access to company resources, but have limited control over the devices used to access them.

One common solution to control devices is to require users to agree to follow a policy.
These policies take the form of user agreements, written in natural language, which describe how the devices should be used and configured.
Various guides are available for companies wishing to implement a policy from governments, standards bodies, and companies seeking to advise~\cite{nicholas_r._c._guerin_security_2008,souppaya_guidelines_????,hp_byod_????,cesg_byod_2015}.
On top of user agreements companies may also use \ac{MDM} software, which can enforce some policies automatically.
The use of \ac{MDM} software does not guarantee compliance however.
One survey found over 50\% of companies with \ac{MDM} software still have uncompliant devices in their networks~\cite{mobileiron_security_labs_q4_2015}.

BYOD policies are becoming more complex and \ac{MDM} software is becoming more proficient.
There has been much work looking at developing the \ac{MDM} software to enforce aspects of BYOD policies~\cite{costantino_towards_2013,martinelli_enhancing_2016,armando_enabling_2014}.
Enforcing a policy is only part of the problem, however.
BYOD policies are specified informally using natural language, and they contain more than just access control decisions.
These policies describe trust relationships inside the company between the IT departments, users, and HR each who may be delegated to make decisions and provide rules.
Policies contain rules that require employees to acknowledge risks, and regulations.
An antivirus or \ac{MDM} program may be used to \emph{implement} part of a policy, but it is the policy which \emph{specifies} which software to use and when; there is no automatic way to check how a policy has been implemented and by what.

This paper describes a formalization of five BYOD policies using AppPAL---a policy language for mobile device privacy preferences~\cite{hallett_apppal_2016}.
Using our formalization we identify the common concerns and trust relationships in these policies.
We show how AppPAL can be used to implement these policies, and describe precisely the different trust relationships.
AppPAL is an instantiation of the SecPAL authorization language~\cite{becker_secpal:_2010} for mobile device policies.
We have found, however, that it is a good fit for other policies surround the mobile ecosystem as well~\cite{hallett_specifying_2016}.

\section{My Idea}
\label{sec:idea}
% 2 pages

\section{The Details}
\label{sec:details}
% 5 pages

\section{Related Work}
\label{sec:related}
% 1-2 pages

\section{Conclusions}
\label{sec:conclusions}
% 1/2 page



\bibliography{paper}{}
\bibliographystyle{plain}
\end{document}

