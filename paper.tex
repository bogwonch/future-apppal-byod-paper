\documentclass{article}

\usepackage{microtype}
\usepackage{amsmath,amssymb,mathtools}
\usepackage{hyperref}
\usepackage{booktabs}

\usepackage{acronym}
\acrodef{MDM}{Mobile Device Management}

\usepackage{listings}
\lstdefinelanguage{AppPAL}{%
  morekeywords={if,inf,says,where,true,false,with,is},
  otherkeywords={can-act-as,can-say},
  sensitive=true,
  morestring=[b]',
  literate={\ inf\ }{{$\infty$}}5
}[keywords,strings]
\newcommand{\listingsize}[0]{\footnotesize}
\lstset{%
  basicstyle=\ttfamily\listingsize,
  keywordstyle=\ttfamily\slshape\listingsize,
  stringstyle=\sffamily\listingsize,
  columns=flexible,
  mathescape,
  tabsize=2,
  showstringspaces=true,
  %numbers=left,
  frame=single,
  breaklines=true,
  numberstyle=\tiny\ttfamily\color{Gray},
  commentstyle=\ttfamily\color{Gray}
}
\newcommand{\code}[2][]{\lstinline[breaklines=true, #1]{#2}}

\usepackage{framed}
\newcommand{\todo}[1]{\marginpar{\begin{framed}\footnotesize\sffamily\raggedright #1\end{framed}}}

\newenvironment{policyrule}[1]{%
  \begin{framed}
      \noindent\textbf{\sffamily #1}:~\itshape
}{
  \end{framed}
}

\title{Capturing Policies for BYOD}
\author{Joseph Hallett and David Aspinall}
\begin{document}
\maketitle

\begin{abstract}
  % 4 Sentences
\end{abstract}

\section{Introduction}
\label{sec:intro}
% Describe the problem
% State my contribution
% 1 page

Employees bring their own devices to work.  
In the past employees might have had a dedicated company device. 
Today around 70\% of companies have a BYOD scheme~\cite{schulze_byod_2016} and, in some fields,
  85\% of staff use their personal devices to look up work-sensitive information~\cite{patel_uk_2015}. 
This creates a challenge for IT departments.
They need to control access to company resources, but have limited control over the devices used to access them.

One common solution to control devices is to require users to agree to follow a policy.
These policies take the form of user agreements, written in natural language, which describe how the devices should be used and configured.
Various guides are available for companies wishing to implement a policy from governments, standards bodies, and companies seeking to advise~\cite{nicholas_r._c._guerin_security_2008,souppaya_guidelines_????,hp_byod_????,cesg_byod_2015}.
On top of user agreements companies may also use \ac{MDM} software, which can enforce some policies automatically.
The use of \ac{MDM} software does not guarantee compliance however.
One survey found over 50\% of companies with \ac{MDM} software still have uncompliant devices in their networks~\cite{mobileiron_security_labs_q4_2015}.

BYOD policies are becoming more complex and \ac{MDM} software is becoming more proficient.
There has been much work looking at developing the \ac{MDM} software to enforce aspects of BYOD policies~\cite{costantino_towards_2013,martinelli_enhancing_2016,armando_enabling_2014}.
Enforcing a policy is only part of the problem, however.
BYOD policies are specified informally using natural language, and they contain more than just access control decisions.
These policies describe trust relationships inside the company between the IT departments, users, and HR each who may be delegated to make decisions and provide rules.
Policies contain rules that require employees to acknowledge risks, and regulations.
An antivirus or \ac{MDM} program may be used to \emph{implement} part of a policy, but it is the policy which \emph{specifies} which software to use and when; there is no automatic way to check how a policy has been implemented and by what.

This paper describes a formalization of five BYOD policies using AppPAL---a policy language for mobile device privacy preferences~\cite{hallett_apppal_2016}.
Using our formalization we identify the common concerns and trust relationships in these policies.
We show how AppPAL can be used to implement these policies, and describe precisely the different trust relationships.
AppPAL is an instantiation of the SecPAL authorization language~\cite{becker_secpal:_2010} for mobile device policies.
We have found, however, that it is a good fit for other policies surround the mobile ecosystem as well~\cite{hallett_specifying_2016}.
Using these papers we identify a set of standard decisions and trust relationships used in BYOD policies.
These are BYOD idioms that are capture decisions at common requirements in BYOD policies.
These give a guide for where future work implementing BYOD tools should focus their efforts to cover more aspects of policies.

\section{My Idea}
\label{sec:idea}
% 2 pages

Corporate BYOD policies are tricky to fully enforce.
Part of their policies are prescriptive:  if you configure your device in this way you will mitigate that threat.
The policies contain more than just configuration, however.
Consider this rule taken from the \emph{Security Policy for the use of handheld devices in corporate environments} by SANS~\cite{nicholas_r._c._guerin_security_2008}:

\newcommand{\textbra}[1]{\ensuremath{\left\langle \text{\sffamily #1} \right\rangle}}
\begin{quote}
  Digital camera embedded on handheld devices \emph{might} be disabled in restricted environments, according to \textbra{COMPANY NAME} risk analysis. 
  In sensitive facilities, information can be stolen using pictures and possibly sent using MMS or E-mail services.

  In high-security facilities such as R\&D labs or design manufacturers, camera MUST be disabled.
  Furthermore, MMS messages should be disabled as well, to prevent malicious users from sending proprietary pictures.
\end{quote}

To implement this policy an \ac{MDM} program, such as IBM's MaaS360 could be used~\cite{_ibm_????}.
This software supports app wrapping (where an app is recompiled to use guarded APIs), and supports security ``checkboxes'' to enable security controls and access control policies.
Using techniques like this would implement the recommendation within the rule, but the rule itself contains more than just configuration to implement.  
It talks of \emph{restricted environments} decided by \emph{company risk analysis}; but how is this communicated to the device?
Does it access the list of restricted environments once, from a server?
Does it understand how the risk analysis was performed?  
Can it decide on the basis of a policy if a location is restricted?
The rule also gives a security objective: \emph{prevent malicious users from sending proprietary pictures}.
The guidelines are given, however, for the specific cases of a compliant user using MMS or email.
This may not be sufficient to stop a sufficiently motivated \emph{malicious} user.

Our approach does not try to implement the policy.
Rather we attempt to capture the goals of the policy rule so that the delegations of trust, tools used, and their configuration is made explicit.
This gives us greater clarity as to which tool is being trusted to implement what policy.
It allows us to see who is being trusted to make which decisions.
It also lets us use automatic tools to uncover aspects of the policy that may be unimplemented~\cite{hallett_specifying_2016}.
Continuing with the example above, this can be expressed in AppPAL as:

\begin{lstlisting}
'company' says 'risk-analysis' can-say 
  Location:L isHighSecurityFacility.

'company' says Device:D mustDisableIn(Location, 'camera')
  if Location isHighSecurityFacility.

'company' says Device:D mustDisableIn(Location, 'mms') 
  if Location isHighSecurityFacility.

'company' says User:U hasSatisfied('proprietary pictures policy')
  if U hasDevice(D),
     D mustDisableIn(Location, 'camera'),
     D mustDisableIn(Location, 'mms'),
     Location isHighSecurityFacility.
\end{lstlisting}

\todo{%
  Why is the apppal better?
  Clearer, explicit, can identify problems automatically.
  
  Explain why I'm trying to build a schema for byod policies and why this is worth a paper.
  
  How is this falsifiable?
  Examples of policy idioms we can't express.
}

\section{The Details}
\label{sec:details}
% 5 pages

\subsection{Changes to AppPAL}

In order to describe BYOD specific conditions we instantiate AppPAL with four kinds of facts: \emph{can, has, is} and \emph{must}.
Like other SecPAL-based instantiations~\cite{becker_framework_2009,aziz_secpal4dsa:_2011} we extend the syntax of facts to support these constructs.
Internally, however, they are treated as standard predicates.

\begin{center}
  \footnotesize\sffamily
  \newcommand{\predicate}[3]{#1 \texttt{#2\textit{#3}}}
  \begin{tabular}{l l}
    \toprule
    Fact                              & Meaning                                         \\
    \midrule
    \predicate{subject}{can}{Action}  & The subject is permitted to perform the action. \\
    \predicate{subject}{has}{Action}  & The subject has performed the action.           \\
    \predicate{subject}{is}{Type}     & The subject is a member of the type.            \\
    \predicate{subject}{must}{Action} & The subject must perform the action.            \\
    \bottomrule
  \end{tabular}
\end{center}

Facts of the \emph{must}-kind represent obligations, actions that should be completed if a particular scenario presents itself.
For these facts we add a rule to ensure the obligation is performed.
This rule should be checked periodically to ensure compliance.
\begin{lstlisting}
$\langle\text{speaker}\rangle$ says $\langle\text{subject}\rangle$ hasSatisfiedObligation$\langle\text{Action}\rangle$
  if $\langle\text{subject}\rangle$ must$\langle\text{Action}\rangle$,
     $\langle\text{subject}\rangle$ has$\langle\text{Action}\rangle$.
\end{lstlisting}

When using AppPAL to describe policies, we also need to describe what domain each variable belongs to.
This is performed through the \emph{is}-kind facts.
AppPAL inherits from SecPAL (and Datalog) a safety condition that all variables in the head of an assertion must be referenced in the body.
This leads to policies which can be noisy to read.
For example in the SANS policy there is a rule that states that devices can only connect to wireless access points owned by the company.
This is written in AppPAL as:
\begin{lstlisting}
'company' says Device canConnectToAP(AP)
  if AP isOwnedByCompany,
     Device isDevice,
     AP isAccessPoint.
\end{lstlisting}
To simplify the presentation we allow a special sugared notation for conditionals expressing the type of a variable (\emph{variable \emph{is}Type}).
Any variable in the head of the rule of the form \texttt{\emph{Type}:Variable} is replaced by the variable and a conditional that \texttt{Variable is\emph{Type}} is added to the body.
The rule can be re-written as shown bellow hiding the typing statements.
\begin{lstlisting}
'company' says Device:D canConnectToAP(AP:X)
  if X isOwnedByCompany.
\end{lstlisting}

\subsection{BYOD Policies}

We analysed five different policies to find common idioms.
The first is the \emph{Security Policy Template: Use of Handheld Devices in a Corporate Environment}, published by the SANS Institute~\cite{nicholas_r._c._guerin_security_2008}.
This is a hypothetical policy published to help companies mitigate the threats to corporate assets caused by mobile devices. 
Companies are expected to modify the document to suit their needs.  
The policy is general; not specific to any particular industry, device, or country's legislation.
The second is taken from Healthcare Information Management System Society~\cite{healthcare_information_and_management_systems_society_mobile_2012}; 
  a US non-profit company trying to improve healthcare through IT.  
It is relatively short but interesting because it contains concerns specific to healthcare scenarios, and is written as a contract the users are agreeing to follow. 
In contrast every other policy we looked at is written as a series of rules users should follow to ensure compliance.
The policy is designed as a sample agreement for a system trying to manage personal mobile devices in a healthcare environment.
The third is taken from an british hospital trust~\cite{kennington_mobiles_2014}, and describes the BYOD scheme used in practice at the hospital. 
Finally two simpler policies are taken from Edinburgh university~\cite{david_williamson_bring_2015} and a company specialising in emergency sirens~\cite{code3pse.org_sample_????}. 
These policies are simpler than the other comprised of much more general rules and only a page or two in length.

\begin{table}\centering\footnotesize\sffamily
  \begin{tabular}{l c c c c c}
    \toprule
                             & {SANS}       & {HiMSS}     & {NHS}       & {Edinburgh} & {Sirens}    \\
    \midrule
    Number of rules          & 33           & 15          & 56          &             & 25          \\
    SecPAL assertions        & 71           & 21          & 58          & 10          & 39          \\
    Policy coverage          & 33 (100$\%$) & 14 (93$\%$) & 40 (71$\%$) &             & 22 (88$\%$) \\
    \midrule
    Delegations              & 23           & 5           & 33          & 2           & 13          \\
    Acknowledgement          &              &             &             &             &             \\
    Partial Acknowledgement  &              &             &             &             &             \\
    \bottomrule             \\
  \end{tabular}
  \label{tab:summary}
  \caption{Summary of the contents of each of the BYOD policies.}
\end{table}

The policies are summarised in \autoref{tab:summary}.
Each policy contains a series of \emph{rules}, which are implemented by one or more \emph{AppPAL assertions}.
The \emph{policy coverage} represents the number of rules that have an AppPAL description attached.
Not all the rules in a policy require an AppPAL translation.
For example rule 9.1 in the NHS policy states:
\begin{policyrule}{NHS}
  Downloading of personal apps onto a corporately issued mobile device should be avoided where possible. 
  The Trust would not encourage staff members to download apps for personal use onto a corporately issued mobile device.
  All staff are reminded that they must adhere to the guidance outlined in the Social Media Policy.
\end{policyrule}
The rule does not prohibit any action, and doesn't require the staff to do anything different.
It does remind staff to be aware of a rule which we could implement with an acknowledegement; a statement that the party has acknowledged a rule.
\begin{lstlisting}
'nhs-trust' says Staff:S mustAcknowledge('avoid-installing-personal-apps').
'nhs-trust' says Staff:S mustAcknowledge('social-media-policy').
\end{lstlisting}
In this case, however, since nothing is required even an acknowledgement seems excessive.

\subsection{BYOD Idioms}
\todo{Describe common idioms and patterns found in each of the policies.  Check the earlier draft for some starting work}



\section{Related Work}
\label{sec:related}
% 1-2 pages

\section{Conclusions}
\label{sec:conclusions}
% 1/2 page



\bibliography{paper}{}
\bibliographystyle{plain}
\end{document}

