\documentclass[a4paper]{article}

\usepackage{microtype}
\usepackage{booktabs}
\usepackage{hyperref}
\usepackage{acronym}
\acrodef{MDM}{mobile device management}
\acrodef{BYOD}{bring your own device}

\usepackage{algorithm}
\usepackage{algpseudocode}

% \apppal{} and Listing formatting
\usepackage{listings}
\lstdefinelanguage{AppPAL}{%
  morekeywords={if,inf,says,where,true,false},
  otherkeywords={can-act-as,can-say},
  sensitive=true,
  morestring=[b]',
  literate={\ inf\ }{{$\infty$}}5
}[keywords,strings]
\lstset{%
  basicstyle=\ttfamily\footnotesize{},
  stringstyle=\sffamily\footnotesize{},
  keywordstyle=\ttfamily\slshape\footnotesize{},
  language=AppPAL,
  columns=flexible,
  frame=single,
  framesep=0.5pt,
  framexleftmargin=2.5pt,
  framexrightmargin=2.5pt,
}

\newcommand{\dotdotdot}[1]{#1}
\newcommand{\etal}[0]{et~al{.}\@}
\newcommand{\apppal}[0]{App{P}{A}{L}}

\title{Case Study Reasoning about BYOD Policies in the Healthcare Sector}
\author{Joseph Hallett and David Aspinall}
\begin{document}
\maketitle

% 4 sentences
\begin{abstract}
  Implementing a \ac{BYOD} policy can be difficult especially in hospitals, where there is a need to balance privacy requirements against the need to treat patients.
  It is common to publish a \ac{BYOD} policy to describe how staff should use their devices in the workplace.
  These policies are written using natural language, and employees are expected to read and follow them: this can be tricky as the policies are often poorly specified.
  We show how \apppal{}, a policy language for mobile app preferences, can be used as a tool for reasoning about policies and how it found issues with BYOD policies used in hospitals and as guidelines.
\end{abstract}

\section{Introduction}
% 1 page
\label{sec:intro}
% 1. Describe the problem
%    - BYOD hard
%    - Hospitals are an interesting case
%    - Need tools for reasoning about policies not enforcing

Employees bringing their own devices to work has become the norm:
  around 70\% of all companies having some form of \ac{BYOD} scheme~\cite{schulze_byod_2016}.
In hospitals the use of personal devices as part of patient care is particularly common.
One survey from the UK found that around 80\% of surgical doctors were willing to use their personal devices for work, with 85\% regularly using them to look up information, and 50\% having medical apps already installed on their phones~\cite{patel_uk_2015}.  Another from the US found similar numbers of physicians regularly using their devices as part of their work~\cite{moyer_managing_2013}.

Several healthcare organisations have started looking into using mobile devices as part of patient care, and found that these devices may be helpful~\cite{prgomet_impact_2009,faulds_feasibility_2016}.  
Organisations, such as the US's HiMMS~\cite{seymour_mobile_2012}, have published guidelines helping hospital administrators implement BYOD policies in their workplaces.  
Additionally general advice for other organisations seeking to implement BYOD policies is available from SANS~\cite{nicholas_r._c._guerin_security_2008}, NIST~\cite{souppaya_guidelines_????}, HP~\cite{hp_byod_????} and the CESG~\cite{cesg_byod_2015} amongst others.
The NHS has even published guidelines for developers seeking to develop apps for use in hospitals~\cite{prgomet_impact_2009}.

Whilst there has been a push for bring \ac{BYOD} and mobile devices into the healthcare industry there has been push back against having mandated security policies automatically enforced.  
The use of \ac{MDM} software to enforce policies can be problematic:
  a survey found that even for companies that use \ac{MDM}, over half of them still have devices in their networks that do not meet their policies~\cite{mobileiron_security_labs_q4_2015}.
In the healthcare industry the circumvention of security controls by physicians is normal, and in some cases taught as good practice~\cite{koppel_workarounds_2015}.
The use of tools to automatically enforce policies in a hospital environment is not always the right approach:
\emph{Moyer} concluded that:
\emph{``Since mobile devices are already being used in hospitals, an approach toward educating the users instead of trying to control the technology may be more practical in the short term''}~\cite{moyer_managing_2013}.
Similarly \emph{Faulds}~\etal~echo this sentiment noting that \emph{``concerns about data protection are legitimate''}, and going on to \emph{``encourage the Information Commissioner’s Office, in consultation with professional medical organisations, to publish guidance specific to the use of BYOD technology''}~\cite{faulds_feasibility_2016}.

The conflict between patient care and data security, as well as \ac{MDM} software being relatively limited, has lead to \ac{BYOD} policies being published as natural language documents that employees are supposed to agree to follow.
Using natural language in policies can create ambiguities, and lead to poorly specified rules.
If a policy is poorly specified and confusing then its subjects may chose to ignore it or misinterpret it's rules.
If we wish to improve our policies we need to be able to reason about them clearly.
We need to be able to identify problems before they happen.

% 2. State your contribution
%    - Standard lexicon of predicates and idioms
%    - Formalisation of several BYOD policies
%    - Identification of problems in said policies
%    - Methods for inferring when \apppal{} policies are problematic

\apppal{} is a policy language for describing mobile app preferences~\cite{hallett_apppal_2016}.
We show how \apppal{} can also be used to formalise \ac{BYOD} policies by translating three \ac{BYOD} policies into \apppal{}.
One is taken from an NHS hospital trust~\cite{kennington_mobiles_2014}. 
Another is from HiMMS published as an example policy to help hospitals design their own policies~\cite{healthcare_information_and_management_systems_society_mobile_2012}.
The third is a  general policy from SANS to help organisations to develop their own \ac{BYOD} schemes~\cite{nicholas_r._c._guerin_security_2008}.
Using these policies we suggest a standard lexicon of BYOD predicates and idioms for using when reasoning about \ac{BYOD} policies. 
With the AppPAL formalisation we show how problems with reachability and redundancy can be identified automatically, and suggest where the policies could be improved.

\section{My Idea}
% 2 pages
\label{sec:idea}

% - Brief intro to apppal but mostly link to existing paper
% - Give an overview of the different policies we're looking at
% - Show differences and similarities between the policies
% - Make sure to tie it all into healthcare, but also be clear this is more general

\section{The Details}
% 5 pages
\label{sec:details}

% - Show the standard predicates and explain how they help to give structure to apppal policies
%   - show how they are tied into the trust relationships with patient care (i.e. consent)
% - Show the consistency checker
%   - Give algorithm and intuition
%   - Show examples where it has found problems
% - Show redundancy checks
%   - Algorithm and intuition
%   - Examples where problems were found



  
\section{Related Work}
% 1-2 pages
\label{sec:related}

\section{Conclusions}
% 0.5 pages
\label{sec:conclusions}

\bibliography{paper}{}
\bibliographystyle{plain}
\end{document}

